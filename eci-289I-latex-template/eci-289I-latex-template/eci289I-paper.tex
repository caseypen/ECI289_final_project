\documentclass[a4paper]{article}
\usepackage{eci289I}
\usepackage[pdftex]{graphicx}
\usepackage{natbib}
\usepackage[labelfont=bf,textfont=sf,labelsep=period]{caption} 
%\RequirePackage{lineno} % for line numbers

\pdfpagewidth=210 true mm
\pdfpageheight=297 true mm

\renewcommand{\rmdefault}{phv} % Arial
\renewcommand{\sfdefault}{phv} % Arial

\bibpunct{[}{]}{;}{a}{,}{,~}

\pagestyle{IEMSSheadings}

% Text to appear in the header of the pages
\IEMSShead{Author1 et al. / A Not Specially Very Long Title}



\title{Multi Robots Task Assignments Problems Solved by Genetic Algorithm}

%Authors Names and Affiliations:  Two spaces below the title, 10 pt
%Arial, Upper and Lower Case, underline author presenting
%paper and  provide his/her email address

\author{\underline{Chen Peng}
\address[A1]{\it{Mechanical and Aerospace Engineering Department,
UC, Davis, US (penchen@ucdavis.edu, )}}}

\begin{document}
%\linenumbers

\begin{abstract}
This term paper mainly uses genetic algorithm to solve the multi robots task assignment(MRTA) problems in two situations. We assume that each robot can go directly from one point to another and the distance is just the euclidean distance between two points. One task only needs one robot in one assignment and all tasks need to be visited in one task execution. In the first situation, we have the same amounts of robots as the tasks (m vs m), while in the second condition, we have less robots than tasks. The objective function of both problems is to find the shortest summary distances of all the robots to go over all task points. 
\end{abstract}
\begin{keyword}
MRTA, genetic algorithm, TSP, mTSP
\end{keyword}

\maketitle


\section{Introduction}

Multi-robot systems (MRS) are a group of robots that are designed
aiming to perform some collective behavior.One of the most challenging problems of MRS is how to optimally assign a set of robots to a set of tasks in such a way that optimizes the overall system performance subject to a set of constraints.This problem is known as Multi-robot Task Allocation (MRTA) problem \cite{MRTASOFA}. In the first situation, we want to match each robot with one particular task so that the total distance is shortest. This is very similar to the classic problem of traveling salesman problem: "Given a list of cities and the distances between each pair of cities, what is the shortest possible route that visits each city exactly once and returns to the origin city?"\cite{wikiwebTSP} In the second condition, we consider less robots to visit more tasks. This problem is very similar to mTSP problem: The multiple traveling salesman problem (MTSP) involves scheduling m \textless 1 salesmen to visit a set of n \textgreater m nodes so that each node is visited exactly once.\cite{sedighpour2012effective} 

\section{m robots m tasks}
In the first situation, 


\subsection{Genetic algorithm model}

%Header of the first page should be as appears in this instructions' paper (right justified). Headers of following pages should be: the name of the authors %(initial(s) and name) / title of the paper (centered). If there are more than two authors write first author et al. If the title is too long, cut the title where %needed and put three dots. Headers text should be in italics 8 pt. Arial.
%No footer should appear. \textbf{Do not} number the pages. Numbering of the pages will be done by the editors of the Proceedings.


\subsection{Small numbers trial}
%Equations should be numbered consecutively as they appear in the
%text with Arabic numerals and should be referred to by their
%numbers only, e.g. (3).  Equations must be typed not hand printed.
%About 5 mm should be left clear above and below each equation.


\subsection{Large numbers solving}

% An example of how to reference a figure using its label:
%, e.g. Figure \ref{fig1}, Figure 2, etc.

% This is how to include a figure
% Note that you can also include vector formats (for example PDF or EPS)

% \begin{figure}[h]
% \centering
% \includegraphics[width=10cm]{Banner13b.jpg}
% \caption{A nice caption text.} 
% \label{fig1}
% \end{figure}


Photographs should only be used if essential to the clarity of the paper.  If used they must be with clear contrast and highly glossed.

\subsection{Complexity of algorithm}
%able numbers and captions appear at the top of the tables. To save space, you can allocate tables and captions only in the right/left hand side of the page, so %that the text will be limited to the other hand side of the page. Tables should be numbered consecutively with Arabic numerals, in the order in which reference %is made to them in the text, e.g. Table 1, Table 2, etc.

%\subsection{Conclusions and Recommendations}

%The real value of a paper is reflected in the nature, soundness
%and clarity of the conclusions, so particular care should be taken
%with this section.

\section{Notation and Units}

%If the paper makes extensive use of symbols or other special
%nomenclature they should be listed and defined under this heading.
%Otherwise, all symbols are to be defined when first used.  All
%units are to be SI (metric).


\section{m robots n tasks(m \textless n)}

\subsection{Local search}

%\underline{Full session papers} should not exceed \textbf{8 A4 camera-ready pages}. Longer papers are allowed at the discretion of session organizers. Please coordinate longer papers with them. All session papers will be independently refereed, and could be accepted or could not be accepted.

\subsection{Robot initial alignment}


\subsection{mTSP solver}

%\section{Presentation of Paper at the Congress}

%\subsection{Slides and Transparencies}
%Projected diagrams are intended to assist the oral presentation of
%a paper and should be prepared specifically for this purpose. They
%should allow only essential information but this should be
%technically correct.

%Line thickness should show clear contrasts.  As a general rule,
%the original drawing should be clearly legible when viewed from a
%distance of six times its longest side. Graphs and curves are
%projected primarily to show tendencies and relationships rather
%than to determine numerical values.  Grid lines should therefore
%be minimised or replaced by simple scales running along the axes.
%The curves themselves should be the prominent feature.

%\subsection{Transparencies}

%Transparencies for overhead projection could be used.

\section*{Acknowledgments}
Any particular assistance out of the ordinary may be acknowledged.  Please, do not use the numeral for the acknowledgements section. It is not necessary from the organisers' point of view to record the permission of the author's organisation to publish the paper or the information contained therein.

\section*{References}


%%%%%%%%%%%%%%%%%%%%%%%%%%%%%%%%%%%%%%%%%%%%%%%%%%%%%%%%%%%%%%%%%
\bibliographystyle{eci289I}
\bibliography{eci289I-refs}
%%%%%%%%%%%%%%%%%%%%%%%%%%%%%%%%%%%%%%%%%%%%%%%%%%%%%%%%%%%%%%%%%

\section*{Appendices}



\end{document}
