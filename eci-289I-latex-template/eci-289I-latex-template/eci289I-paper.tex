\documentclass[a4paper]{article}
\usepackage{eci289I}
\usepackage[pdftex]{graphicx}
\usepackage{natbib}
\usepackage[labelfont=bf,textfont=sf,labelsep=period]{caption} 
%\RequirePackage{lineno} % for line numbers

\pdfpagewidth=210 true mm
\pdfpageheight=297 true mm

\renewcommand{\rmdefault}{phv} % Arial
\renewcommand{\sfdefault}{phv} % Arial

\bibpunct{[}{]}{;}{a}{,}{,~}

\pagestyle{IEMSSheadings}

% Text to appear in the header of the pages
\IEMSShead{Author1 et al. / A Not Specially Very Long Title}



\title{Paper Title: 18 pt Arial Bold, Upper and Lower Case}

%Authors Names and Affiliations:  Two spaces below the title, 10 pt
%Arial, Upper and Lower Case, underline author presenting
%paper and  provide his/her email address

\author{\underline{A. Author}
\address[A1]{\it{A affiliation,
GPO Box 123,
Somewhere, SomeCountry (AuthorA@email.zz, AuthorC@email.zz)}}, B.
Author\address[B1]{\it{B affiliation,
GPO Box 987,
Somewhere else, SomeCountry (AuthorB@email.zz, AuthorD@email.zz)}}, C.
Author\addressmark[A1] and D.
Author\addressmark[B1]}

\begin{document}
%\linenumbers

\begin{abstract}
The abstract should be self-contained and explicit, setting out
the ground covered and the principal conclusions reached and
should be \textbf{one paragraph}.  The suggested length is 300 words.
The abstract must start 2 cm below Authors Affiliations and span
the full width of the page.  The format is 10 pt Arial,
fully justified.
\end{abstract}
\begin{keyword}
Start keywords one space below the abstract and provide 3 to 5
keywords separated by semicolons.
\end{keyword}

\maketitle


\section{Body of Paper}

Main headings are Caps, Bold.  Text must flow in one column spreading the text over the full width of the page.  All
text is 10 pt Arial, fully justified. Leave 2 spaces above
and one space below headings and sub-headings

\subsection{Introduction}

Secondary headings (sub-headings) are Upper and Lower Case, Bold.
Second level sub-headings are Upper and Lower Case, Italics, and its use it is not recommended.

The main purpose of an introduction is to enable the paper to be understood without undue reference to other sources.  It should therefore have sufficient background material for this purpose.  Generally, highly specialised papers will not need an extensive introduction as interested readers may be expected to be familiar with current literature on the subject.  On the other hand, when a paper is likely to interest people working in fields outside the immediate area of the paper, the introduction should contain background material which could otherwise be scattered throughout the literature.

\subsection{Headers and Footers}

Header of the first page should be as appears in this instructions' paper (right justified). Headers of following pages should be: the name of the authors (initial(s) and name) / title of the paper (centered). If there are more than two authors write first author et al. If the title is too long, cut the title where needed and put three dots. Headers text should be in italics 8 pt. Arial.
No footer should appear. \textbf{Do not} number the pages. Numbering of the pages will be done by the editors of the Proceedings.


\subsection{Equations}
Equations should be numbered consecutively as they appear in the
text with Arabic numerals and should be referred to by their
numbers only, e.g. (3).  Equations must be typed not hand printed.
About 5 mm should be left clear above and below each equation.


\subsection{Figures and Photographs}
Figures must be of high quality and can be in colour for the conference proceedings. Figure numbers and captions appear at the foot of the figures. To save space, you can allocate figures and captions only in the right/left hand side of the page, so that the text will be limited to the other hand side of the page. Figures should be numbered consecutively with Arabic numerals, in the order in which reference is made to them in the text.

% An example of how to reference a figure using its label:
%, e.g. Figure \ref{fig1}, Figure 2, etc.

% This is how to include a figure
% Note that you can also include vector formats (for example PDF or EPS)

% \begin{figure}[h]
% \centering
% \includegraphics[width=10cm]{Banner13b.jpg}
% \caption{A nice caption text.} 
% \label{fig1}
% \end{figure}


Photographs should only be used if essential to the clarity of the paper.  If used they must be with clear contrast and highly glossed.

\subsection{Tables}
Table numbers and captions appear at the top of the tables. To save space, you can allocate tables and captions only in the right/left hand side of the page, so that the text will be limited to the other hand side of the page. Tables should be numbered consecutively with Arabic numerals, in the order in which reference is made to them in the text, e.g. Table 1, Table 2, etc.

\subsection{Conclusions and Recommendations}

The real value of a paper is reflected in the nature, soundness
and clarity of the conclusions, so particular care should be taken
with this section.

\section{Notation and Units}

If the paper makes extensive use of symbols or other special
nomenclature they should be listed and defined under this heading.
Otherwise, all symbols are to be defined when first used.  All
units are to be SI (metric).


\section{Criteria for Acceptance}

\subsection{Length and Other Details}

\underline{Full session papers} should not exceed \textbf{8 A4 camera-ready pages}. Longer papers are allowed at the discretion of session organizers. Please coordinate longer papers with them. All session papers will be independently refereed, and could be accepted or could not be accepted.

\subsection{Permission to Publish}
Unless informed by the author to the contrary, the Society will
assume that a paper submitted has not been published or offered
elsewhere and is not the property of any other person or body.

It is the author's responsibility to obtain any necessary
permission from his/her organisation or from any other person or
body for the publication of a paper or any material in it; such
permission need not be mentioned in the acknowledgments.

\section{Presentation of Paper at the Congress}

\subsection{Slides and Transparencies}
Projected diagrams are intended to assist the oral presentation of
a paper and should be prepared specifically for this purpose. They
should allow only essential information but this should be
technically correct.

Line thickness should show clear contrasts.  As a general rule,
the original drawing should be clearly legible when viewed from a
distance of six times its longest side. Graphs and curves are
projected primarily to show tendencies and relationships rather
than to determine numerical values.  Grid lines should therefore
be minimised or replaced by simple scales running along the axes.
The curves themselves should be the prominent feature.

\subsection{Transparencies}

Transparencies for overhead projection could be used.

\subsection{Powerpoint or Similar Presentation}

Equipment for electronic projection on a beamer will be available.

\section{Registration}

At least one of the authors should be registered for final
acceptance of the paper and its inclusion in the Congress
Proceedings.


\section*{Acknowledgments}
Any particular assistance out of the ordinary may be acknowledged.  Please, do not use the numeral for the acknowledgements section. It is not necessary from the organisers' point of view to record the permission of the author's organisation to publish the paper or the information contained therein.

\section*{References}

Do not use the numeral for the references section.

\subsection*{Style}
References should include (in the following order): Author Name(s), Initials, Title of article with first letter uppercase, full Journal name in Italics, Volume (Number), page range, date.  The page range must be hyphenated. A 4 mm indentation must be left for each reference. An example of a reference in a conference paper is given below.

\subsection*{Order}

The references must be listed in alphabetical order of author's
names and increasing dates of publication, with the addition of an
'a' or 'b' to the date, where necessary.  In the text reference is
made to writing the surname of the author, followed by the date of
publication in square brackets, e.g. "it was shown by
\citet{hanke:dfwudc} that ...". Where more than two authors are
involved, the reference in the text should be of the form: "it was
shown by \citet{churchman:itor}".

It was shown by \citet{berry:teomowcihbsi} that...

%%%%%%%%%%%%%%%%%%%%%%%%%%%%%%%%%%%%%%%%%%%%%%%%%%%%%%%%%%%%%%%%%
\bibliographystyle{eci289I}
\bibliography{eci289I-refs}
%%%%%%%%%%%%%%%%%%%%%%%%%%%%%%%%%%%%%%%%%%%%%%%%%%%%%%%%%%%%%%%%%

\section*{Appendices}
If more than one, appendices should be lettered A, B, etc., e.g.
Appendix A.




\end{document}
